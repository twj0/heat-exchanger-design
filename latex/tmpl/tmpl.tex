% !TeX program = xelatex
\documentclass[12pt,a4paper,fontset=windows]{ctexart}

% 版心与段落
\usepackage[left=2.5cm,right=2.5cm,top=2.8cm,bottom=2.8cm]{geometry}
\usepackage{setspace}
\setstretch{1.25}  % 对应说明里的"多倍 1.25 倍行距"

\setCJKfamilyfont{myhei}{SimHei}[AutoFakeBold=2]
\renewcommand{\heiti}{\CJKfamily{myhei}}

\usepackage{titlesec}
% 一级/二级/三级标题黑体字号:小三 / 四号 / 小四
\titleformat{\section}{\heiti\fontsize{15pt}{18pt}\selectfont}{\thesection\ }{0.5em}{}
\titleformat{\subsection}{\heiti\fontsize{12pt}{14pt}\selectfont}{\thesubsection\ }{0.5em}{}
\titleformat{\subsubsection}{\heiti\fontsize{10.5pt}{13pt}\selectfont}{\thesubsubsection\ }{0.5em}{}
\titlespacing*{\section}{0pt}{0.8ex plus .2ex}{0.6ex}
\titlespacing*{\subsection}{0pt}{0.6ex}{0.4ex}
\titlespacing*{\subsubsection}{0pt}{0.5ex}{0.3ex}

% 段落缩进与间距
\usepackage{indentfirst}
\setlength{\parindent}{2em}

% 数学与编号(实现 "3-2-1" 的公式编号样式)
\usepackage{amsmath,amssymb}
\numberwithin{equation}{subsection}
\renewcommand{\theequation}{\thesection-\arabic{subsection}-\arabic{equation}}

% 图表
\usepackage{graphicx}
\usepackage{booktabs}
\usepackage{array}
\usepackage{multirow}
\usepackage{tabularx}
\usepackage{longtable}

% 参考文献(示意,按需要换成 biblatex/bst)


% 超链接
\usepackage[hidelinks]{hyperref}

% —— 下划线工具(可控厚度、留白长度)——
\usepackage[normalem]{ulem} % 提供 \uline
\ULdepth = 0.8ex
\renewcommand\ULthickness{.6pt}
% 可控长度的下划线填空:\fillin{标签}{5cm}
\newcommand{\fillin}[2]{\makebox[#2]{\uline{\hspace*{#2}}}\hspace{-#2}\makebox[#2][l]{#1\hfill}}

% 统一小四号/五号快速命令
\newcommand{\xiaosi}{\fontsize{12pt}{16pt}\selectfont}
\newcommand{\wuhao}{\fontsize{10.5pt}{14pt}\selectfont}

% 封面样式
\newcommand{\Cover}{
  \begin{titlepage}
    \centering
    {\heiti\fontsize{18pt}{22pt}\selectfont 东莞理工学院化学工程与能源技术学院\par}
    \vspace{12pt}
    {\heiti\fontsize{22pt}{26pt}\selectfont 换热器课程设计\par}
    \vspace{6pt}
    {\heiti\fontsize{20pt}{24pt}\selectfont 报告书\par}
    \vspace{24pt}

    \begin{tabular}{@{}p{5em}p{26em}@{}}
    \wuhao 设计题目: & \uline{\hfill \hfill \hfill \hfill \hfill} \\
    \wuhao 专\hspace{1.2em}业: & 能源与动力工程 \\
    \wuhao 班\hspace{1.2em}级: & 2023能源动力1班 \\
    \wuhao 学生姓名: & wjTang \\
    \wuhao 学\hspace{1.2em}号: & 2023428020130 \\
    \wuhao 指导教师: & 胡冰 \\
    \wuhao 年\hspace{1.2em}月\hspace{1.2em}日: & \uline{\hfill \hfill \hfill \hfill \hfill} \\
    \end{tabular}

    \vfill
  \end{titlepage}
}

% "课程设计任务书"区块(含下划线填空与分周进度)
\newenvironment{TaskSheet}{
  \section*{课程设计任务书}
  \addcontentsline{toc}{section}{课程设计任务书}
  \wuhao
  \begin{tabular}{@{}p{8em}p{26em}@{}}
  专业: & 能源与动力工程 \\
  班级: & 2023能源动力1班 \\
  学生姓名: & wjTang \\
  学号: & 2023428020130 \\
  指导教师: & 胡冰 \\
  设计题目: & \uline{\hfill \hfill \hfill \hfill \hfill} \\
  \end{tabular}

  \vspace{0.6em}
  \noindent\heiti 设计目的、主要内容(参数、方法)及要求\par
}{
}

% —— 评分标准表:稳健、可复制、无花哨线型,跨页兼容 —— 
% 宽度列型
\newcolumntype{Y}{>{\wuhao\centering\arraybackslash}p{6.5em}}
\newcolumntype{M}[1]{>{\wuhao\centering\arraybackslash}m{#1}}
\newcolumntype{L}[1]{>{\wuhao\raggedright\arraybackslash}m{#1}}

\newenvironment{RubricTable}{
\begin{table}[htbp]
\centering
\begin{tabular}{|M{8em}|L{9em}|Y|Y|Y|Y|M{6em}|M{5em}|}
}{
\end{tabular}
\end{table}
}

\begin{document}

% 封面
\Cover

% 任务书
\begin{TaskSheet}
\textbf{设计目的:}\\[-0.2em]
培养综合运用所学知识、查阅换热器资料获取有关知识和数据、进行换热设备初步设计的能力。培养独立工作及发现问题、分析问题、解决问题的能力。提高计算能力、培养工程实际观念。熟悉掌握常用软件的使用。培养认真的学风和工作作风。

\vspace{0.6em}
\textbf{主要内容:}\\[-0.2em]
根据峰谷电电价差异及蓄热应用,设计一种新型的换热器设备并进行相应的理论分析和计算。

\vspace{0.6em}
\textbf{要求:}
\begin{enumerate}\wuhao
\item 确定应用地区(峰谷电差价);
\item 具体场合(建筑、小区;进出水/蒸汽温度;流量;供热量);
\item 换热器形式(何种类型换热器结构、二维或三维图);
\item 换热器尺寸,热量衡算(温度,面积,换热系数等);
\item 经济性分析;
\item 总结;
\item 发挥能力,争取做到具有一定的应用和创新。
\end{enumerate}

\vspace{0.6em}
\textbf{进度安排}\\[-0.2em]
\wuhao
~ 第1周:\ \uline{\hfill\hfill\hfill\hfill\hfill}\par
第2--3周:\ \uline{\hfill\hfill\hfill\hfill\hfill}\par
第4--5周:\ \uline{\hfill\hfill\hfill\hfill\hfill}\par
第6--10周:\ \uline{\hfill\hfill\hfill\hfill\hfill}\par
第11周:\ \uline{\hfill\hfill\hfill\hfill\hfill}\par
第12--14周:\ \uline{\hfill\hfill\hfill\hfill\hfill}\par

\vspace{0.6em}
\textbf{主要参考资料}\ \uline{\hfill\hfill\hfill\hfill\hfill}\par

\vspace{1.2em}
\hfill 指导教师签字:\ \uline{\hspace{6em}}
\end{TaskSheet}

% 摘要与关键词
\section*{摘\hspace{1em}要}
\addcontentsline{toc}{section}{摘要}
\xiaosi(中文摘要小四号宋体,段首缩进2字,1.5倍/或1.25倍行距按需调整。摘要精炼,交代研究对象、方法、结论与意义。)

\vspace{0.5em}
\noindent\textbf{\xiaosi 关键词:} \xiaosi 关键词1,关键词2,关键词3,关键词4,关键词5

% 目录
\clearpage
{\heiti\fontsize{16pt}{19pt}\selectfont 目\hspace{1em}录}\par
\vspace{0.5em}
\tableofcontents
\clearpage

% 正文示意(严格按照模板小节结构)
\section{课程设计的定义和目的}
\subsection{课程设计的定义}
课程设计是“针对某一门”课程的要求,对学生进行综合性训练的过程,其中包括参考资料的查找、相关工具的应用以及课程设计文本的撰写和设计的实现或仿真等。

\subsection{课程设计的目的}
课程设计的目的在于培养学生运用课程中所学到的理论知识,解决实际问题的能力,培养学生查阅资料文献的能力,培养学生使用相关软件的能力,培养学生动手的能力,培养学生规范撰写的能力等。

\section{课程设计准备工作}
孔子曰“工欲善其事,必先利其器。”
\subsection{相关文献、书目的准备}
(此处照模板说明撰写)
\subsection{相关工具软件的准备}
(此处照模板说明撰写)

\section{课程设计撰写}
\subsection{摘要的格式}
(一级标题排版、中文摘要/关键词小四宋体,英文摘要 Times New Roman 小四,1.5 倍行距,关键词 3--5 个,用逗号分隔)
\subsection{标题及正文文字的格式}
(正文小四宋体;段首缩进 2 字;行距多倍 1.25)
\subsection{目录格式}
(用大纲级别自动生成;目录上方“目录”二字,黑体三号)
\subsection{数学表达式的格式}
示例:
\begin{equation}
E = mc^2
\end{equation}
上式编号样式为“3-2-1”,右对齐引用同上。
\subsection{图、表的格式}
(图题:五号宋体加粗,图与正文空一行;表题置于表上方、顶格、五号宋体加粗)

% —— 课程设计报告评分标准(完整可分页表格)——
\clearpage
\section*{课程设计报告评分标准}
\addcontentsline{toc}{section}{课程设计报告评分标准}

\begin{RubricTable}
\hline
教学目标要求 & 评分标准 & 90--100 & 80--89 & 60--79 & 0--59 & 权重(\%) & 评分 \\
\hline
目标1:学习态度与主动性 &
在具体的工作实际中,能够认真对待自己所承担的相应任务,表现出极佳的学习态度和学习能力。能够根据下达的课程设计任务,主动构思设计方案,确定设计流程,制定设计目标,能够积极地与小组成员进行问题的讨论、分析并提出初步的解决方法。 &
\checkmark & & & & 20 & \\
\cline{2-8}
目标1:学习态度与主动性 &
在具体的工作实际中,能够认真对待自己所承担的相应任务,表现出较好的学习态度和学习能力。能够根据下达的课程设计任务,构思设计方案,确定设计流程,制定设计目标,能够与小组成员进行问题的讨论、分析并提出初步的解决方法。 & & \checkmark & & & & \\
\cline{2-8}
目标1:学习态度与主动性 &
在具体的工作中,较认真地完成了相关课程设计任务,学习态度较好,能够根据下达课程设计任务,参与构思设计方案,制定设计目标。 & & & \checkmark & & & \\
\cline{2-8}
目标1:学习态度与主动性 &
在整个课程设计的过程中,学习态度较差,缺乏学习主动性,出勤率较低。在下达课程设计任务之后,不能主动制定设计目标,对自己承担的任务认识较为混乱。 & & & & \checkmark & & \\
\hline

目标2:方案与分析能力 &
在设计方案的讨论过程中,表现出良好的思考能力和逻辑分析能力,对问题的解决表现出强烈的愿望。设计合理、理论分析与计算正确,文献查阅能力强、引用合理、调查调研非常合理、可信。 &
\checkmark & & & & 25 & \\
\cline{2-8}
目标2:方案与分析能力 &
在设计方案的讨论过程中,表现出较好的思考能力和逻辑分析能力,对问题的解决表现出良好的主动性。设计合理、理论分析与计算正确,文献引用、调查调研比较合理、可信。 & & \checkmark & & & & \\
\cline{2-8}
目标2:方案与分析能力 &
在设计方案的论证过程中,能够提出自己的建议,能够针对课程设计中遇到的相关问题提出解决方法。设计合理,理论分析与计算基本正确,主要文献引用、调查调研比较可信。 & & & \checkmark & & & \\
\cline{2-8}
目标2:方案与分析能力 &
在下达课程设计任务之后,不能主动制定设计目标,对自己承担的任务认识较为混乱。设计不合理,理论分析与计算有原则错误,文献引用、调查调研有较大的问题。 & & & & \checkmark & & \\
\hline

目标3:实验与动手能力 &
实验数据准确,有很强的实际动手能力和计算机应用能力。 &
\checkmark & & & & 30 & \\
\cline{2-8}
目标3:实验与动手能力 &
实验数据比较准确,有较强的实际动手能力和计算机应用能力。 & & \checkmark & & & & \\
\cline{2-8}
目标3:实验与动手能力 &
实验数据比较准确,有一定的实际动手能力。 & & & \checkmark & & & \\
\cline{2-8}
目标3:实验与动手能力 &
实验数据不可靠,实际动手能力差。 & & & & \checkmark & & \\
\hline

目标4:报告与图纸规范性 &
最终的设计报告结构严谨,逻辑性强,层次清晰,语言准确,文字流畅,完全符合规范化要求,书写工整或用计算机打印成文;图纸非常工整、清晰。 &
\checkmark & & & & 25 & \\
\cline{2-8}
目标4:报告与图纸规范性 &
最终的设计报告结构合理,符合逻辑,文章层次分明,语言准确,文字流畅,符合规范化要求,书写工整或用计算机打印成文;图纸工整、清晰。 & & \checkmark & & & & \\
\cline{2-8}
目标4:报告与图纸规范性 &
最终的设计报告结构合理,层次较为分明,文理通顺,基本达到规范化要求,书写比较工整;图纸比较工整、清晰。 & & & \checkmark & & & \\
\cline{2-8}
目标4:报告与图纸规范性 &
最终的设计报告内容空泛,结构混乱,文字表达不清,错别字较多,达不到规范化要求;图纸不工整或不清晰。 & & & & \checkmark & & \\
\hline
\end{RubricTable}

\vspace{0.8em}
\noindent \wuhao 指导教师签名:\ \uline{\hspace{8em}} \hfill 指导教师评定成绩:\ \uline{\hspace{8em}}

% 参考文献(示例)
\clearpage
\section*{参 考 文 献}
\addcontentsline{toc}{section}{参 考 文 献}
\wuhao
[1] 王化成. 高级财务管理学[M]. 北京: 中国人民大学出版社, 2003: 15--18\par
[2] 巩亦平. 中小企业财务管理中存在的问题及对策[J]. 财会研究, 2005, (3): 16--20

% 附录
\clearpage
\section*{附\hspace{1em}录}
\addcontentsline{toc}{section}{附 录}
\wuhao 课程设计中的程序如下:\par
\bigskip
\noindent\framebox[\textwidth]{\parbox{0.96\textwidth}{\wuhao 在此粘贴或附程序/代码等附录材料。}}

\end{document}