\documentclass{article}

\usepackage[utf8]{ctex}
\usepackage[a4paper, margin=1in]{geometry}
\usepackage{tabularray} % 加载 tabularray
\usepackage{booktabs} % tabularray 内部可以很好地集成 booktabs 的命令

\begin{document}

\pagestyle{plain}

\section*{课程设计报告评分标准}
\addcontentsline{toc}{section}{课程设计报告评分标准}

% 使用 longtblr 环境创建可跨页表格
\begin{longtblr}[
  caption = {课程设计报告评分标准},
  label = {tab:evaluation_criteria},
]{
  width = \textwidth, % 表格总宽度
  colspec = {
    Q[l, m, 2.5cm]  % 第1列:左对齐,垂直居中,宽度2.5cm
    X[j]             % 第2-5列:两端对齐,自动计算宽度
    X[j] 
    X[j] 
    X[j] 
    Q[c, m, 1.5cm]  % 第6列:居中对齐,垂直居中,宽度1.5cm
    Q[c, m, 1.5cm]  % 第7列:居中对齐,垂直居中,宽度1.5cm
  },
  hlines, vlines, % 添加所有内外框线
  rowhead = 2, % 重复前两行作为表头
  row{1} = {font=\bfseries}, % 第一行加粗
  row{2} = {font=\bfseries}, % 第二行加粗
}
% --- 表头 ---
\SetCell[r=2]{c} 教学目标要求 & \SetCell[c=4]{c} 评分标准 & & & & \SetCell[r=2]{c} 权重(\%) & \SetCell[r=2]{c} 评分 \\
& 90-100 & 80-89 & 60-79 & 0-59 & & \\

% --- 表格内容 ---
\textbf{目标1:} & 在具体的工作实际中,能够认真对待自己所承担的相应任务,表现出极佳的学习态度和学习能力。能够根据下达的课程设计任务,主动构思设计方案,确定设计流程,制定设计目标,能够积极地与小组成员进行问题的讨论、分析并提出初步的解决方法。 & 在具体的工作实际中,能够认真对待自己所承担的相应任务,表现出较好的学习态度和学习能力。能够根据下达的课程设计任务,构思设计方案,确定设计流程,制定设计目标,能够与小组成员进行问题的讨论、分析并提出初步的解决方法。 & 在具体的工作中,较认真地完成了相关课程设计任务,学习态度较好,能够根据下达的课程设计任务,参与构思设计方案,制定设计目标。 & 在整个课程设计的过程中,学习态度较差,缺乏学习主动性,出勤率较低。在下达课程设计任务之后,不能主动制定设计目标,对自己承担的任务认识较为混乱。 & 20 & \\

\textbf{目标2:} & 在设计方案的讨论过程中,表现出良好的思考能力和逻辑分析能力,对问题的解决表现出强烈的愿望。设计合理、理论分析与计算正确,文献查阅能力强、引用合理、调查调研非常合理、可信。 & 在设计方案的讨论过程中,表现出较好的思考能力和逻辑分析能力,对问题的解决表现出良好的主动性。设计合理、理论分析与计算正确,文献引用、调查调研比较合理、可信。 & 在设计方案的论证过程中,能够提出自己的建议,能够针对课程设计中遇到的相关问题提出解决方法。设计合理,理论分析与计算基本正确,主要文献引用、调查调研比较。 & 在下达课程设计任务之后,不能主动制定设计目标,对自己承担的任务认识较为混乱。设计不合理,理论分析与计算有原则错误,文献引用、调查调研有较大的问题。 & 25 & \\

\textbf{目标3:} & 实验数据准确,有很强的实际动手能力和计算机应用能力。 & 实验数据比较准确,有较强的实际动手能力和计算机应用能力。 & 实验数据比较准确,有一定的实际动手能力。 & 实验数据不可靠,实际动手能力差。 & 30 & \\

\textbf{目标4:} & 最终的设计报告结构严谨,逻辑性强,层次清晰,语言准确,文字流畅,完全符合规范化要求,书写工整或用计算机打印成文;图纸非常工整、清晰。 & 最终的设计报告结构合理,符合逻辑,文章层次分明,语言准确,文字流畅,符合规范化要求,书写工整或用计算机打印成文;图纸工整、清晰。 & 最终的设计报告结构合理,层次较为分明,文理通顺,基本达到规范化要求,书写比较工整;图纸比较工整、清晰。 & 最终的设计报告内容空泛,结构混乱,文字表达不请,错别字较多,达不到规范化要求;图纸不工整或不清晰。 & 25 & \\

% --- 【新增】指导教师签名行 ---
\textbf{指导教师签名} & & \SetCell[c=6]{l} \textit{指导教师评定成绩} & & & & & \\

\end{longtblr}

\end{document}